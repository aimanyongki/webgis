\documentclass[12pt]{report}
\usepackage{graphicx}
\usepackage{hyperref}
\usepackage{amsmath,bm}
\usepackage{verbatim}
\usepackage[english]{babel}

\begin{document}
\pagenumbering{arabic}

\newpage

\chapter*{Web mapping sederhana}
Setelah mengetahui cara menginstall GeoServer yang telah dijelaskan pada bagian \texttt{README.md}, pada penjelasan ini akan dibahas mengenai penggunaan GeoServer untuk menampilkan shape file.   

\section*{Creating a new workspace}
Setelah berhasil membuka \texttt{http://localhost:8080/geoserver/web/}, tahapan selanjutnya adalah \newline

1. Download shape file pada laman ini \url{https://data.biogeo.ucdavis.edu/data/diva/adm/IDN_adm.zip}. \\

2. Ekstrak data dan pindahkan semua data ke folder \url{GEOSERVER_DATA_DIR/data/}. \\

3. Pada panel data sebelah kiri pilih Workspace dan kemudian \textbf{Add new workspace}. \\

4. Masukkan misalnya Nama = \texttt{shpindo}  dan Nama URI = \url{http://geoserver.org/shpindo}. \\

5. Klik \textbf{Submit}. Workspace \texttt{shpindo} telah ditambahkan ke daftar Workspace. \\

\section*{Creating a store}
Store akan membuat GeoServer mengetahui bagaimana untuk terhubung dengan shapefile.\newline 

1. Pada panel \textbf{Data} pilih \textbf{Stores}.\\

2. Add \textbf{new Store} kemudian pilih Shapefile.\\

3. Tambahkan informasi yang sesuai dengan data yang kita masukkan.\\

4. Pada \textbf{Connection Parameters}, pilih lokasi dari shapefile.\\

5. Klik \textbf{Save}.

\section*{Creating a layer} 
1. Pada laman \textbf{New Layer}, klik \textbf{Publish}. \\

2. Masukkan informasi yang sesuai yang mendefinisikan Layer.\\

3. Untuk menghasilkan bounding boxes dari layer, pilih \textbf{Compute from data} dan kemudian \textbf{Compute from native bounds}.\\

4. Klik \textbf{Publishing} tab.\\

5. Set layer style. Pada WMS Settings, pastikan \textbf{Default Style} pada pilihan Line.\\

6. Save 

\section*{Previewing the layer}
1. Pada \textbf{Layer Preview}, klik \textbf{OpenLayers} yang tertulis pada tab \textbf{Common Formats}.\\

2. Peta \textbf{OpenLayers} akan ditampilkan pada tab baru dari browser.
- Peta OpenLayers akan ditampilkan pada tab baru dari browser.  

\end{document}
